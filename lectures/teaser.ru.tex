\documentclass[aspectratio=169,notopbar]{beamer}
\usepackage[OT2,T2A]{fontenc}
\usepackage[utf8]{inputenc}
\usepackage[russian,english]{babel}
\usepackage{ferres}

\framelogo{img/pymc-labs-logo}
%math
\input{math_commands.tex}
\author{Максим Кочуров}
\title[Байес в действии, ЦУ 2025]{Байес в действии}
\institute{Центральный Университет}
\date{}

\begin{document}
\begin{frame}
	\maketitle
\end{frame}
\begin{frame}{Обо мне}
\begin{columns}
\begin{column}{0.8\linewidth}
\begin{itemize}
    \item Образование:
        \begin{itemize}
            \item бакалавриат -- ЭФ МГУ (2018)
            \item магистратура - Skoltech DS (2020)
        \end{itemize}
    \item Один из разработчиков PyMC
    \item Ведущий дата сайентист в PyMC Labs
    \item Выпускник BayesGroup
    \item Имею опыт работы с:
    \begin{itemize}
        \item машинным обучением и компьютерным зрением;
        \item дифференциальной геометрией для графовых нейросетей;
        \item байесовскими методами для глубокого обучения;
        \item прикладной байесовский статистикой в индустрии (А/Б тестирование, биоинформатика).
    \end{itemize}
\end{itemize}
\end{column}
\begin{column}{0.2\linewidth}
\includegraphics[width=\linewidth]{img/msu-logo}
\includegraphics[width=\linewidth]{img/sk-logo}
\includegraphics[width=\linewidth]{img/bayesgroup-logo}
\includegraphics[width=\linewidth]{img/pymc-logo}
\includegraphics[width=\linewidth]{img/pymc-labs-logo}
\end{column}
\end{columns}
\end{frame}
\section{Тёмный лес}%The Woods}
\begin{frame}{Обзор задач}
\includegraphics[width=\linewidth]{img/Problems Landscape.jpeg}
\end{frame}
\begin{frame}{Думайте о задаче}
    \begin{columns}
    \begin{column}{0.6\linewidth}
    Классическое машинное обучение имеет ограничения.%Classical Machine learning gives a lot of constraints
    \begin{itemize}
        \item Мало контроля параметризации моделей%Model parametrization is out of control
        \item Сложно добавить иерархию%Hierarchies are hard to add
        \item Гибкость -- не приоритет%Flexibility is never priority
    \end{itemize}
    Фокус на \textbf{алгоритмах}, не на решаемой задаче %Не факт: например, свёрточные нейросети предназначены именно для решения задач с локальными структурами (например, изображения), а рекуррентные -- для анализа последовательностей.
    %Thinking is \textbf{algorithm focused}, not the problem focused
    \begin{block}{Оговорка}
    Обычно это не проблема, но не все задачи решаются стандартными способами.
    %Which is not usually a problem, but not every problem is solved in a standard way.
    \end{block}
    \end{column}
    \begin{column}{0.4\linewidth}
    \includegraphics[width=0.8\linewidth]{img/lock.jpeg}
    \end{column}
    \end{columns}
\end{frame}
\begin{frame}{Думайте о задаче}
    \begin{columns}
    \begin{column}{0.6\linewidth}
    "White box" модели более гибкие
    %White box models allow flexibility
    \begin{itemize}
        \item Контроль параметризации%Model parametrization is in direct control
        \item Структура модели -- Ваше решение%Нейросети тоже можно как угодно структурировать, добавлять туда решатели диффуров, расшаривать параметры, skip connections и пр.
        %Model structure is your decision
        \item Возможности покажутся магией%What is possible will sound magic
    \end{itemize}
    Фокус на \textbf{задаче}, алгоритмы -- лишь инструмент.
    %Thinking is \textbf{problem focused}, algorithms are complimentary.
    \begin{block}{Оговорка}
    Удостоверьтесь, что Ваша задача не типа "fit-predict". В противном случае от байесовских методов не будет дополнительных плюсов.
        %Make sure your problem is not "fit-predict", otherwise there is no added value from Bayesian methods.
    \end{block}
    \end{column}
    \begin{column}{0.4\linewidth}
    \includegraphics[width=\linewidth]{img/galaxy.png}
    \end{column}
    \end{columns}
\end{frame}
\begin{frame}{Идентификация задачи}
\begin{columns}
\begin{column}{0.6\linewidth}
Данные обычно структурированы%Data is usually structured
\begin{itemize}
\item Вложенные/иерерхические%Nested / hierarchical data
\item Временные ряды%Time-series data
\item Прочая нетривиальная структура%Other non-trivially structured data sets
\end{itemize}
Ваша задача -- не просто предсказание:%Your problem is not a simple prediction problem:
\begin{itemize}
    \item Если Вам нужны только предсказания, ML для Вас%If you care about predictions only, ML is for you
    \item Если Вам не нужна интерпретация, DL для Вас%If you do not need interpretation, DL is for you
\end{itemize}
\end{column}
\begin{column}{0.4\linewidth}
\includegraphics[width=\linewidth]{img/glass.png}
\end{column}
\end{columns}
\end{frame}
\section{Expanding Dimensions}
\subsection{Hierarchy}
\begin{frame}{Иерархия}
    \begin{columns}
        \begin{column}{0.5\linewidth}
            \begin{itemize}
                \item Прогноз объёма прода для нескольких магазинов%Predicting Sales for Multiple Stores
                % Bayesian hierarchical modeling allows us to model the variation in sales within each store as well as the variation between different stores. By taking into account the hierarchical structure of the data, we can make more accurate sales predictions and improve inventory management.
                \item Анализ распространения инфекционных заболеваний%Understanding the Spread of Infectious Diseases
                % Bayesian hierarchical modeling can be used to model the spatial and temporal variation in disease incidence across multiple locations. By considering the hierarchical structure of the data (e.g., cases nested within counties or states), we can make more accurate predictions of disease spread and identify the factors that contribute to higher rates of transmission.
                \item Прогноз потребительских предпочтений%Predicting Consumer Preferences
                % Bayesian hierarchical modeling enables us to model the variation in consumer preferences within individual households, while also considering the overall patterns of consumer behavior across different households. This approach can help identify factors that drive consumer behavior and inform marketing strategies.
            \end{itemize}
        \end{column}
        \begin{column}{0.5\linewidth}
            \includegraphics[width=0.9\linewidth]{img/Hierarchy.jpeg}
        \end{column}
    \end{columns}
\end{frame}
\begin{frame}{Используем весь потенциал иерархий}
\begin{columns}
    \begin{column}{0.5\linewidth}
            \begin{itemize}
                \item<alert@1> Неравные группы вносят вклад в общее априорное распределение%Unequal groups contribute to population prior
                \item<alert@2> Байесовский вывод: распределение знания об общей структуре между группами%Bayesian inference shares knowledge between groups
                \item<alert@3> Можно симулировать ненаблюдаемые группы%You can simulate unobserved groups from population
            \end{itemize}
        \end{column}
        \begin{column}{0.5\linewidth}
            \only<1>{\includegraphics[width=0.8\linewidth]{img/population-prior-1.jpeg}}
            \only<2>{\includegraphics[width=0.8\linewidth]{img/population-prior-2.jpeg}}
            \only<3>{\includegraphics[width=0.8\linewidth]{img/population-prior-3.jpeg}}
        \end{column}
    \end{columns}
\end{frame}
\begin{frame}{Прогноз для ненаблюдаемых групп}
    \begin{columns}
    \begin{column}{0.5\linewidth}
            \begin{itemize}
                \item<alert@1> Вы рработаете на нескольких региональных рынках и хотите расширяться%You have regional markets and want to expand
                \item<alert@2> Нужно угадать предпочтения людей в новых регионах%You need to guess new region preferences
                \item<alert@3> Используйте байесовские методы, чтобы симулировать ненаблюдаемое%Use Bayesian Methods to simulate the Unseen
            \end{itemize}
        \end{column}
        \begin{column}{0.5\linewidth}
            \only<1>{\includegraphics[width=0.8\linewidth]{img/europe-1.jpeg}}
            \only<2>{\includegraphics[width=0.8\linewidth]{img/europe-2.jpeg}}
            \only<3>{\includegraphics[width=0.8\linewidth]{img/europe-3.jpeg}}
        \end{column}
    \end{columns}
\end{frame}
\begin{frame}{Бонус: иерархии в географии}
    \begin{columns}
    \begin{column}{0.5\linewidth}
            \begin{itemize}
                %\item Geography is gold
                \item Группы взаимозависимы%Groups depend on each other
                \item Можно использовать эту взаимозависимость, чтобы легко повысить качество предсказаний%You can use it to boost your prediction with ease
            \end{itemize}
        \end{column}
        \begin{column}{0.5\linewidth}
            \includegraphics[width=\linewidth]{img/geography-gold.png}
        \end{column}
    \end{columns}
\end{frame}
\section{Априорные распределения}
\begin{frame}{Априорные распределения -- это круто}%Priors are Hot}
    \begin{columns}
    \begin{column}{0.5\linewidth}
            \begin{itemize}
                \item Позволяют отделить зёрна от плевел%????? %Separate the wheat from the chaff
                \item %Tell what you think of of your problem
                \item Let the model take the expertise seriously 
            \end{itemize}
        \end{column}
        \begin{column}{0.5\linewidth}
            \includegraphics[width=\linewidth]{img/relevant.jpeg}
        \end{column}
    \end{columns}
\end{frame}
\begin{frame}{Когда мало данных}
    \begin{columns}
    \begin{column}{0.6\linewidth}
    \begin{itemize}
        \item Остаётся опираться на экспертные знания.%With few data all you have is your expert knowledge
        \item Можно вложить эти знания в априорные распределения%Putting it into the priors makes the model usable
        \item Каждый пример важен%Every example matters
    \end{itemize}
        \end{column}
        \begin{column}{0.4\linewidth}
            \includegraphics[width=\linewidth]{img/reality-uncertainty.jpeg}
        \end{column}
    \end{columns}
\end{frame}
\begin{frame}{Структура модели: причинность}
\begin{columns}
    \begin{column}{0.62\linewidth}
    \begin{itemize}
        \item Не давайте "чёрному ящику" принимать решения%Do not let the black box model decide
        \item Задайте собственную структуру%Give your own structure
        \item Структуры разные: "одна на всех" не подойдёт%One size does not fit all
    \end{itemize}
        \end{column}
        \begin{column}{0.38\linewidth}
            \includegraphics[width=\linewidth]{img/causality.jpeg}
        \end{column}
    \end{columns}
\end{frame}


\begin{frame}{p-значение в А/Б тестах}
\textbf{Что из нижеперечисленного верно?}
\begin{columns}
    \begin{column}{0.7\linewidth}
    \begin{enumerate}
        \item $p$ -- вероятность того, что верна нулевая гипотеза.
        \item $(1-p)$ -- вероятность того, что верна альтернативная гипотеза.
        \item $p \le 0.05$ значит, что нулевая гипотеза неверна и дожна быть
отвергнута.
        \item Значение $p > 0.05$ значит, что эффект отсутствует.
    \end{enumerate}
    \end{column}
    \begin{column}{0.3\linewidth}
        \includegraphics[width=\linewidth]{img/p-value-1}
    \end{column}
\end{columns}
    \pause
    \begin{alertblock}{Использовать p-значение плохо?}
    Я не предлагаю отказаться от p-значений, я предлагаю лучше понять
их%I do not urge you to give up p-values, but I urge you to add more understanding.
    \end{alertblock}
\end{frame}

\begin{frame}{Интерпретация p-значений}
%Greatest insights into p-values:
\begin{figure}
    \centering
    \includegraphics[width=0.8\linewidth]{img/confidence-intervals}
\end{figure}
\end{frame}
\begin{frame}{Неопределённость}
\begin{columns}
    \begin{column}{0.5\linewidth}
        \begin{align*}
        \text{ROI} &= \frac{\text{NPV}-\text{CapEx}}{\text{CapEx}}\\
        \text{NPV} &= (\text{CLV}(r) - \text{OpEx}(r)) \cdot \Delta N
    \end{align*}
    \end{column}
    \begin{column}{0.5\linewidth}
        \begin{itemize}
            \item CLV(r) - Discounted User Value
            \item OpEx(r) - Operational Cost
        \end{itemize}
    \end{column}
    \end{columns}
    \vspace{1em}
\begin{columns}
    \begin{column}{0.6\linewidth}
    \begin{itemize}
        \item В финансах нужно оценивать риски%In finance you need risks
        \item Делайте симуляции, не доверяйте точечным оценкам%Simulate, do not trust point estimates
        \item Байес помогает принимать информированные решения%Bayes helps to take informed decisions
    \end{itemize}        
    \end{column}
    \begin{column}{0.4\linewidth}
        \includegraphics[width=\linewidth]{img/roi-plot.png}
    \end{column}
\end{columns}
\end{frame}
\begin{frame}{Кейс: AB(C) тест}
\begin{columns}
    \begin{column}{0.5\linewidth}
        \textbf{Проблемы}
        \begin{itemize}
            \item Работает пр равных (50/50) объёмах выборок
            \begin{itemize}
                \item Сложнее при объёмах 10/90%Complicated to do 10/90
            \end{itemize}
            \item Точечные оценки размера эффекта%Point estimates for effect size
            \begin{itemize}
                \item Сложно действовать на их основе%Hard to turn into actions
            \end{itemize}
            \item Бутстрэп делать долго%Time consuming Bootstrap
            \begin{itemize}
                \item Плюс он не очень помогает, когда мало данных%Does not help much on small sample sizes
            \end{itemize}
            \item Предполагаем простые распределения%Assumes simple distributions
        \end{itemize}
    \end{column}
    \begin{column}{0.5\linewidth}
        \includegraphics[width=\linewidth]{img/p-value.png}
    \end{column}
\end{columns}
\end{frame}
\begin{frame}{Кейс: AB(C) тест}
    \begin{columns}
        \begin{column}[t]{0.3\linewidth}
        \begin{block}{Сложные наблюдения}
            Тестирование сразу 2-х гипотез, не-нормальное распределение
            %Testing two hypotheses at once, Non-Normal likelihood
        \end{block}
        \begin{itemize}
            \item Объём покупок\\%Purchase size\\
            \includegraphics[height=1.5cm]{img/nut}
            \item Вероятность покупки\\
            \includegraphics[height=0.5cm]{img/nut}
            \includegraphics[height=0.5cm]{img/nut}
            \includegraphics[height=0.5cm]{img/nut}
        \end{itemize}
        \end{column}
        \hspace{1em}
        \begin{column}[t]{0.3\linewidth}
        \begin{block}{Множественные тесты}
            Нет статистических поправок%No statistical corrections
        \end{block}
        \begin{itemize}
            \item 3 или 10 тестов!%Can run 3 or 10 tests!
            \item Пробные тесты на малых выборках!%Conduct pilot tests on small samples!
            \item Объёмы выборок могут быть не сбалансированы!%Unbalanced sample sizes are possible!
            \item Тестируем в онлайн режиме!%Complete the test in online mode!
        \end{itemize}
        \end{column}
        \hspace{1em}
        \begin{column}[t]{0.3\linewidth}
        \begin{block}{Экономика}
            Можно вычислить экономический эффект реализации%Can calculate the economic effect of implementation
        \end{block}
        \begin{itemize}
            \item Какая самая плохая окупаемость инвестиций (ROI)?%What is the worst return on Investment?
        \end{itemize}
        \end{column}
    \end{columns}
\end{frame}
\begin{frame}{Новые результаты: линейная регрессия}
    \begin{columns}
        \begin{column}{0.6\linewidth}
            \begin{enumerate}
                \item Стандартизируйте данные: $a\mapsto \frac{a-\operatorname{mean}(a)}{\operatorname{std}(a)}$
                \item Decide on $R^2$
                \item Decide on feature importance
                \item Decide on effect size sign
                \item Done
            \end{enumerate}
        \end{column}
        \begin{column}{0.4\linewidth}
        \begin{figure}
            \centering
            \includegraphics[width=0.5\linewidth]{img/qr-linear-regressions.png}\\
            Webinar about \\ Awesome Linear Regressions
        \end{figure}
            
        \end{column}
    \end{columns}
    \begin{block}{Even more formulas}
        This is a recently developed the R2D2M2 prior\cite{aguilar2023intuitive}, read more detailed math there. And PyMC implementation of the approach R2D2M2CP (the extension) \cite{pymc-experimental-r2d2m2cp}.
    \end{block}
\end{frame}
\begin{frame}{Кейс: маркетинг}
    \begin{figure}
    \centering
        \includegraphics[height=2cm]{img/vk-logo}
        \includegraphics[height=2cm]{img/tiktok-logo}
        \includegraphics[height=2cm]{img/fb-logo}
        \includegraphics[height=2cm]{img/instagram-logo}
        \includegraphics[height=2cm]{img/twitter-logo}\\
        \includegraphics[height=2.5cm]{img/ads-icon}
        \includegraphics[height=2.5cm]{img/MMM-logo}
    \end{figure}
\begin{block}{\textbf{M}edia \textbf{M}ix \textbf{M}odel}
    MMM -- помогает оценивать маркетинговые каналы по историческим
данным%helps to evaluate marketing channels from historical data
\end{block}
\end{frame}

\begin{frame}{Кейс: маркетинг}
\begin{columns}
    \begin{column}{0.5\linewidth}
    \begin{itemize}
        \item Априорные распределения: устанавливаем стоимость привлечения клиентов%Priors: Setting Customer Acquisition Costs
        \item Структура: эта стоимость меняется%Structure: Acquisition costs vary
        \item Неопределённость: выделяем бюджет обоснованно%Uncertainty: Allocate budget thoughtfully
    \end{itemize}
    \begin{block}{Результаты}
        \begin{itemize}
            \item Лучшее понимание каналов%Better understanding of channels
            \item Эффективное привлечение клиентов%Efficient customer acquisition
        \end{itemize}
    \end{block}
    \end{column}
    \begin{column}{0.5\linewidth}
        \includegraphics[width=\linewidth]{img/cac-time-varying.png}
        \includegraphics[height=1cm]{img/pymc-labs-logo.png}\hspace{1em}\includegraphics[height=1cm]{img/hello-fresh-logo.png}
    \end{column}
\end{columns}
\end{frame}

\begin{frame}{Кейс: выращиваем зерно как в Ceres}
    \begin{columns}
    \begin{column}{0.5\linewidth}
    \begin{itemize}
        \item Как оптимизировать объём удобрений?%How to optimize for fertilizer?
        \item Как оптимизировать для региона?%How to optimize for region?
        \item Как оптимизировать для обоих?%How to optimize for both?
    \end{itemize} 
    \begin{block}{Решение}
        Объёмное А/Б тестирование%Extensive AB testing required
    \end{block}
    \end{column}
    \begin{column}{0.5\linewidth}
        \includegraphics[width=0.8\linewidth]{img/Ceres.jpeg}
        \includegraphics[height=1cm]{img/pymc-labs-logo.png}\hspace{1em}\includegraphics[height=1cm]{img/indigo-logo.png}
    \end{column}
\end{columns}
\end{frame}

\begin{frame}{Кейс: выращиваем зерно как в Ceres}
    \begin{columns}
    \begin{column}{0.5\linewidth}
    \begin{itemize}
        \item Иерархия: общий эффект для регионов%Hierarchies: Include region common effect
        \item Структура: соседние поля взаимосвязаны%Structure: Close fields interact
        \item Прогноз: урожайность нового поля%Predictions: Estimate yield on a new field
    \end{itemize} 
    \begin{block}{Результат}
        Пайплайн для экспериментов с удобрениями на уровне континента%Established pipeline to experiment with fertilizers on the continent level
    \end{block}
    \end{column}
    \begin{column}{0.5\linewidth}
    \centering
        \includegraphics[width=0.8\linewidth]{img/crops.jpeg}
        \includegraphics[height=1cm]{img/pymc-labs-logo.png}\hspace{1em}\includegraphics[height=1cm]{img/indigo-logo.png}
    \end{column}
\end{columns}
\end{frame}

\begin{frame}{Хотите изучить байесовские методы?}
    \begin{columns}
    \begin{column}{0.3\linewidth}
    \textbf{Мы начинаем в сентябре 2025}
    \end{column}
    \begin{column}{0.3\linewidth}
    \includegraphics[width=\linewidth]{img/update-your-prior.jpeg}
    \end{column}
    \begin{column}{0.3\linewidth}
    
    \end{column}
\end{columns}
\end{frame}
\begin{frame}[allowframebreaks]
        \frametitle{Библиография}
        \bibliographystyle{abbrv}
        \bibliography{references.bib}
\end{frame}
\end{document}

\documentclass{beamer}
\usepackage[notopbar]{ferres}
\framelogo{img/msu-logo}

%math
\input{math_commands.tex}
\author{Max Kochurov}
\title[Practical Bayes - Bayesian Thinking]{Bayesian Thinking}
\institute[MSU]{Moscow State University}
\date{Lecture 1}
\begin{document}
\begin{frame}
	\maketitle
\end{frame}

\begin{frame}{Agenda}
\tableofcontents
\end{frame}

\section{About Me}
\begin{frame}{About Me}
\begin{columns}
\begin{column}{0.75\linewidth}
\begin{itemize}
    \item Graduated from 
        \begin{itemize}
            \item BS - MSU EF (2018)
            \item MS - Skoltech DS (2020)
        \end{itemize}
    \item Core developer at PyMC
    \item Principle Data Scientist at PyMC Labs
    \item BayesGroup alumni
    \item Experience with 
    \begin{itemize}
        \item Large scale Deep Learning and Computer Vision
        \item Differential geometry for Graph Neural Networks
        \item Bayesian Methods for Deep Learning
        \item Applied Bayesian Statistics for industry (e.g. AB testing, Bio-Informatics)
    \end{itemize}
\end{itemize}
\end{column}
\begin{column}{0.25\linewidth}
\includegraphics[width=\linewidth]{img/msu-logo}
\includegraphics[width=\linewidth]{img/sk-logo}
\includegraphics[width=\linewidth]{img/bayesgroup-logo}
\includegraphics[width=\linewidth]{img/pymc-logo}
\includegraphics[width=\linewidth]{img/pymc-labs-logo}
\end{column}
\end{columns}
\end{frame}
\section{Formal stuff}
\begin{frame}{In this course}
You'll learn...
\begin{itemize}
    \item how to think critically about your model
    \item tools to check the validity of the results
    \item how to present your results
    \item non-parametric models for time series 
\end{itemize}
\end{frame}
\begin{frame}{Grading}
The grade consists of
    \begin{itemize}
        \item 60\% Homework
        \item 40\% Group project
    \end{itemize}
Grades will be assigned as
    \begin{itemize}
        \item 5 - $85\%+$
        \item 4 - $65\%+$
        \item 3 - $40\%+$
        \item 2 - $<40\%$
    \end{itemize}
\end{frame}
\section{Motivation}
\begin{frame}{Why learn Bayesian methods?}
    \begin{itemize}
        \item Used in advanced research
        \begin{itemize}
            \item CBR - \href{https://www.cbr.ru/Content/Document/File/119374/bDSGE.pdf}{Bayesian DSGE (link)}
            \item Papers using PyMC from google scholar is \href{https://scholar.google.com/scholar?hl=ru&as_sdt=2005&sciodt=0,5&cites=6936955228135731011&scipsc=&q=&scisbd=1}{overwhelming (link)}
        \end{itemize}

        \item Used in top notch industry
        \begin{itemize}
            \item \href{https://engineering.hellofresh.com/bayesian-media-mix-modeling-using-pymc3-for-fun-and-profit-2bd4667504e6}{Marketing at Indigo (link)}
            \item \href{https://pubmed.ncbi.nlm.nih.gov/27442271/}{Drug development at Roche (link)}
            \item \href{https://github.com/quantopian/bayesalpha}{Portfolio Theory at Quantopian (link)}
            \item \href{https://support.everysk.com/hc/en-us/articles/1500001040721-Private-Investments}{Financial Advisory at Everysk (link)}
            \item \href{https://discourse.pymc.io/t/job-opening-director-of-data-science-at-civiqs/1895}{Conducting Surveys at Civiqs (link)}
        \end{itemize}
        \item Growth opportunity
        \begin{itemize}
            \item Links many disciplines and career transitions
            \item Hot non-boring job offerings in industry
            \item Opens new research possibilities
        \end{itemize}
    \end{itemize}
\end{frame}
\section{Bayesian Probability}
\begin{frame}{Bayesian Probability}

    \center {\Huge $\underbrace{p(\Theta|\gD)}_{\text{\includegraphics[width=2cm]{img/breaking-news}}}
    =\frac{
        \overbrace{p(\gD|\Theta)}^{\text{\includegraphics[width=2cm]{img/fact-label}}}\overbrace{p(\Theta)}^{\includegraphics[width=2cm]{img/thinker-statue}}
        }{
        p(\gD)
        }$}
    $$\gD = \text{Data}\qquad \Theta = \text{World State}$$
\end{frame}
\subsection{Prior}
\begin{frame}{Prior Distribution}
    \center \Huge $\textcolor{gray}{p(\Theta|\gD)
    =}\frac{
        \textcolor{gray}{p(\gD|\Theta)}\overbrace{p(\Theta)}^{\includegraphics[width=2cm]{img/thinker-statue}}
        }{
        \textcolor{gray}{p(\gD)}
        }$
\end{frame}
\begin{frame}{Case Study: Where do priors come from?}
Authors: Marielle Zondervan-Zwijnenburg, Margot Peeters, Sarah Depaoli, Rens van de Schoot \cite{where-do-priors-come-from}. Bayesian Econometrics example
\begin{block}{Policy question}
    Should we increase cannabis control for adolescents?
\end{block}
\begin{itemize}
    \item<+-> Drugs long term influence on brain activity after early onset
    \item<+-> Few to zero relevant prior research
    \begin{itemize}
        \item no exact match for cannabis case and brain activity
        \item additional developing diseases 
    \end{itemize}
    \item<+-> Scarce, hard to obtain data
    \item<+-> Classical econometrics fails (16 data points in the
    group)
    \item<+-> Expert knowledge feels important
    \item<+-> Statistics is required for an informed decision
\end{itemize}
\end{frame}
\begin{frame}{The Game}
    \begin{figure}
        \centering
        \includegraphics[width=\linewidth]{img/sopt}
        \caption{Sample picture from self ordered pointing test}
        \label{fig:my_label}
    \end{figure}
\end{frame}
\begin{frame}{Quick intro}
Measuring the existence and severity of cannabis usage
    \begin{itemize}
        \item You can measure brain development with a Game ( Self Ordered Pointing Test \cite{sopt})
        \item Cannabis use is checked for participants
        \item Adolescents pass the Game 2 times a year
        \item The results are compared between the \textbf{heavy} and \textbf{light} cannabis users
    \end{itemize}
\begin{block}{Leftover questions}
\begin{itemize}
    \item How does the amount of cannabis used affects the brain development?
    \item What age is sufficient to minimize the effect of usage?
    \item What policy should be used to minimize the effect?
\end{itemize}        
\end{block}

\end{frame}

\begin{frame}{Case Study: Prior Distribution}
\begin{columns}
\begin{column}{0.8\linewidth}
To develop a prior researchers combined many sources of information
    \begin{enumerate}
        \item Knowledge before seeing any data
        \visible<2->{
        \begin{itemize}
            \item Practical range for SOPT measure and growth rates
        \end{itemize}}
        \item Prior research results
        \visible<3->{
        \begin{itemize}
            \item effect size of cannabis use was mixed with other diseases or had missing information
            \item "13 out of 693 articles yielded useful information"
        \end{itemize}}
        \item Expert knowledge
        \visible<4->{
        \begin{itemize}
            \item Prior research was reviewed by experts
            \item Relation of diseases and behaviour to cannabis use
        \end{itemize}}
        \item Constraints
        \visible<5->{
        \begin{itemize}
            \item Slope (SOPT development rate) is more positive than negative
        \end{itemize}}
        \item Model properties
        \visible<6->{
        \begin{itemize}
            \item Sign of the quadratic term
        \end{itemize}}
    \end{enumerate}
\end{column}
\begin{column}{0.2\linewidth}
\includegraphics[width=\linewidth]{img/emoji/see-no-evil-monkey}
\includegraphics[width=\linewidth]{img/emoji/hear-no-evil-monkey}
\includegraphics[width=\linewidth]{img/emoji/speak-no-evil-monkey}
\end{column}
\end{columns}
\end{frame}
\begin{frame}{Results}
    \begin{figure}
        \centering
        \includegraphics[width=0.8\linewidth]{img/sopt-reprodusing}
    \end{figure}
\end{frame}

\begin{frame}{Common Issues}
In the research the prior was defended
\begin{enumerate}
    \item Prior is subjective
    \visible<2->{
    \begin{itemize}
        \item Informed prior was compared to uninformed one
    \end{itemize}
    }
    \item Prior specification is unclear
    \visible<3->{
    \begin{itemize}
        \item The log book was provided with a full description of the choice
        \item \url{https://osf.io/aw8fy/}
    \end{itemize}
    }
    \item Prior is incorrectly specified
    \visible<4->{
    \begin{itemize}
        \item There are still some issues with the analysis but it is out of the scope
    \end{itemize}
    }
\end{enumerate}

\end{frame}
\subsection{Likelihood}
\begin{frame}{Likelihood Distribution}
    \center \Huge $\textcolor{gray}{p(\Theta|\gD)
    =}\frac{
        \overbrace{p(\gD|\Theta)}^{\text{\includegraphics[width=2cm]{img/fact-label}}}{p(\Theta)}
        }{
        \textcolor{gray}{p(\gD)}
        }$
\end{frame}

\begin{frame}{Case Study: AB Test}
\begin{columns}
    \begin{column}{0.5\linewidth}
        You sell nuts. You want sell more nuts!
        How to increase sales?
        \begin{itemize}
            \item increase purchase probability
            \visible<2->{
            \begin{itemize}
                \item Create a banner about healthy food
                \item Add a banner with recipes
                \item Improve the layout
            \end{itemize}
            }
            \item increase order size
            \visible<3->{
            \begin{itemize}
                \item Lower the price
                \item Increase quality
                \item Make better packaging
            \end{itemize}
            }
        \end{itemize}
    \visible<4->{
        \begin{block}{What is better?}
            AB testing can answer the question
        \end{block}
    }
    \end{column}
    \begin{column}{0.5\linewidth}
    \begin{figure}
        \centering
        \includegraphics[width=\linewidth]{img/nuts}
        \includegraphics[width=0.4\linewidth]{img/chek}
        \includegraphics[width=0.55\linewidth]{img/stonks}
    \end{figure}
    \end{column}
\end{columns}
\end{frame}
\begin{frame}{Not all customers buy nuts}
\begin{figure}
    \centering
    \foreach \n in {1,...,6}{%
\includegraphics[width=0.049\linewidth]{img/chek}\hspace{0.001\linewidth}%
\includegraphics[width=0.049\linewidth]{img/chek}\hspace{0.001\linewidth}%
\hspace{0.05\linewidth}%
}%
\includegraphics[width=0.049\linewidth]{img/chek}\hspace{0.001\linewidth}%
\end{figure}
\begin{itemize}
    \item A significant portion of data are just zeros
    \item A classical 2 sample t-test assumes normality, not our case
    \item Researchers admit t-test weaknesses in these cases \cite{when-t-test-fails}
\end{itemize}
\pause
\begin{block}{Solution}
    Think of a non-normal likelihood
\end{block}
\end{frame}
\begin{frame}{Zero inflation}
\begin{columns}
\begin{column}{0.7\linewidth}
\begin{block}{Zero Inflation}
        Data property, when a significant portion of data is exactly zero
    \end{block}

\end{column}
\begin{column}{0.3\linewidth}
\includegraphics[width=\linewidth]{img/binary}
\end{column}
\end{columns}
    
    Examples:
    \begin{itemize}
        \item Wait times in a queue (no queue is zero)
        \item Defects on a production line (no defects is zero)
        \item Rain water level (sunny weather is 0 water level)
        \item Purchase order statement (no nuts is zero)
    \end{itemize}
\end{frame}
\begin{frame}{Zero Inflated Distribution}
    \begin{block}{Wisdom}
        Any distribution can be made zero inflated
    \end{block}
    \textbf{Example:} Zero Inflated Gamma.
    
    We'll have parameters $\alpha, \beta$ (from Gamma) and $p$ - probability of non-zero
    
    \textbf{Sampling:}
    \begin{align*}
        z &\sim \operatorname{Bernoulli}(p)\\
        sample &\sim \begin{cases}
        \operatorname{Gamma}(\alpha, \beta)&, z = 1\\
        0&, z =0
        \end{cases}
    \end{align*}
    \pause
    \textbf{Log Probability Density Function}
    \begin{equation*}
        \log p(x\cond p, \alpha, \beta) = \begin{cases}
        \log(1-p) &,x=0\\
        \log(p) + \alert<3->{\frac {x^{\alpha -1}e^{-\beta x}\beta ^{\alpha }}{\Gamma (\alpha )}}
        &,x> 0
        \end{cases}
    \end{equation*}
\end{frame}
\begin{frame}{Zero inflation as a mixture}
    \begin{block}{Wisdom}
        Zero inflation is a special case of a Mixture
    \end{block}
    Components:
    \begin{enumerate}
        \item $\operatorname{Constant}(0)$
        \item $\operatorname{Gamma}(\alpha, \beta)$
    \end{enumerate}
    Mixture probability is $p$
    \begin{equation*}
        \operatorname{ZI-Gamma}(p, \alpha, \beta) \equiv \operatorname{Mixture}([1-p, p], [\operatorname{Constant}(0), \operatorname{Gamma}(\alpha, \beta)])
    \end{equation*}
\end{frame}
\begin{frame}{Back to the example}
    \begin{align*}
        \text{make an order} &\sim \operatorname{Bernoulli}(p) \\
        \text{order amount} &\sim \begin{cases}
        \operatorname{Gamma}(\alpha, \beta)&, \text{make an order} = 1\\
        0&, \text{make an order} =0
        \end{cases}
    \end{align*}
\begin{columns}
\begin{column}{0.7\linewidth}
\includegraphics[height=3cm]{img/person}
\includegraphics[height=3cm]{img/shop}
\end{column}
\begin{column}{0.3\linewidth}
\includegraphics[height=1.5cm]{img/nut}\includegraphics[height=0.5cm]{img/nut}\includegraphics[height=0.5cm]{img/nut}\includegraphics[height=0.5cm]{img/nut}\\
\includegraphics[height=1.5cm]{img/nut-crossed}
\end{column}
\end{columns}
\end{frame}
\begin{frame}{Takeouts}
    \begin{itemize}
        \item Good likelihood helps to get better sense of the problem
        \begin{itemize}
            \item split purchase probability and purchase amount
            \item more possibilities over a classical t-test
        \end{itemize}
        \item Understanding a problem is a first step to a good likelihood
    \end{itemize}
\end{frame}
% \begin{frame}{How AB test is connected to likelihood?}
%     \begin{columns}
%     \begin{column}{0.5\linewidth}
%     \textbf{Classical AB test is 2 sample t-test}
%     \begin{enumerate}
%         \item Data in each group are normally distributed
%         \item Data values are continuous
%         \item The variances for the two independent groups are equal
%         \item For very small groups of data, it can be hard to test these requirements
%     \end{enumerate}
%     \end{column}
%     \begin{column}{0.5\linewidth}
%     \textbf{Bayesian AB test}
%     \begin{enumerate}
%         \item Data follows any known distribution
%         \item Data values may be both continuous or discrete
%         \item The variances for the two independent groups may be any
%         \item Any sample size is acceptable
%     \end{enumerate}
%     \end{column}
%     \end{columns}
%     \begin{block}{Takeout}
%         Classical approach limits us in applications (likelihood choice)
%     \end{block}
% \end{frame}
\subsection{Posterior}
\begin{frame}{Posterior Distribution}
\center \Huge $\underbrace{p(\Theta|\gD)}_{\text{\includegraphics[width=2cm]{img/breaking-news}}}
    =\frac{
        \textcolor{gray}{p(\gD|\Theta)p(\Theta)}
        }{
        \textcolor{gray}{p(\gD)}
        }$
\end{frame}
\begin{frame}{Posterior Distribution}
\begin{columns}
\begin{column}{0.75\linewidth}
\begin{align*}
&p(\text{what you think}|\text{data})\\
\propto &p(\text{data} |\text{what you think})p(\text{what you think})
\end{align*}
\end{column}
\begin{column}{0.25\linewidth}
\includegraphics[width=\linewidth]{img/meme/faith-meme}
\end{column}
\end{columns}
\end{frame}
\begin{frame}{Case Study: Marketing}
    \begin{figure}
    \centering
    \includegraphics[height=1cm]{img/dollar}\includegraphics[height=1cm]{img/dollar}\includegraphics[height=1cm]{img/dollar}\\
        \includegraphics[height=2cm]{img/vk-logo}
        \includegraphics[height=2cm]{img/tiktok-logo}
        \includegraphics[height=2cm]{img/fb-logo}
        \includegraphics[height=2cm]{img/instagram-logo}
        \includegraphics[height=2cm]{img/twitter-logo}\\
        \includegraphics[height=2.5cm]{img/ads-icon}
        \includegraphics[height=2.5cm]{img/MMM-logo}
    \end{figure}
\begin{block}{\textbf{M}edia \textbf{M}ix \textbf{M}odel}
    MMM - helps to evaluate marketing channels from historical data
\end{block}
\end{frame}
\begin{frame}{Uncertainty}
\begin{columns}
\begin{column}{0.5\linewidth}
Media Mix Models (\href{https://engineering.hellofresh.com/bayesian-media-mix-modeling-using-pymc3-for-fun-and-profit-2bd4667504e6}{read more here})
    \begin{itemize}
        \item How valuable is additional \$1000 invested in VK? Or Facebook?
        \item How certain is the model estimation?
        \item High value, high uncertainty or low value low uncertainty?
        \item How to allocate money?
    \end{itemize}
\end{column}
\begin{column}{0.5\linewidth}
\includegraphics[width=\linewidth]{img/high-value-high-uncertainty}
\end{column}
\end{columns}
\visible<2>{
\begin{block}{Takeout}
Uncertainty helps to make more informed decision
\end{block}
}
\end{frame}
\begin{frame}{Recap}
    In Bayesian framework you have:
    \begin{itemize}
        \item Prior = What I think the problem is
        \item Likelihood = What the facts I have
        \item Posterior = What the problem actually is \textit{given priors and data}
    \end{itemize}
\end{frame}
\section{Model}
\begin{frame}{The Model}
    \center
    { \Huge $p_\gM(\Theta|\gD)
    =\frac{
        p_\gM(\gD|\Theta)p_\gM(\Theta)
        }{
        p_\gM(\gD)
        }$}
\vfill
\begin{block}{}
Treat the model as a "one of many" descriptions of the problem.
\end{block}
\end{frame}
\begin{frame}{Bias - Variance trade-off}
\begin{columns}
\begin{column}{0.4\linewidth}
Getting to a good model
\begin{enumerate}
    \item Start with a over-simplified model
    \item Make sure it samples well
    \item Increase the complexity
    \item ...
    \item Choose the best model using cross validation
\end{enumerate}
\end{column}
\begin{column}{0.6\linewidth}
\includegraphics[width=\linewidth]{img/bias-variance}
\end{column}
\end{columns}
\visible<2->{
\begin{block}{Common mistake}
Starting with a complicated model make debugging painful
\end{block}
}
\end{frame}
\begin{frame}{Case Study: generalized additive models}
\begin{equation*}
    y(t)= \underbrace{g(t)}_{\text{trend}} + \underbrace{s(t)}_{\text{seasonality}} + \underbrace{r(X_t)}_{\text{regressors}} + \varepsilon_t
\end{equation*}
\begin{columns}
\begin{column}{0.5\linewidth}
Adding more and more complexity
\begin{enumerate}
    \item start with a simple trend model
    \item add seasonality
    \item add fine seasonality details, holidays or other features
\end{enumerate}
\end{column}
\begin{column}{0.5\linewidth}
\includegraphics[width=\linewidth]{img/airline_passengers}
\end{column}
\end{columns}
\end{frame}
\section{Discussion}
\begin{frame}{When Bayes?}
Bayesian modeling does not fit all the use cases at once
\begin{itemize}
    \item It requires extra skills compared to classical machine learning
    \item You might not need the extra "uncertainty"
\end{itemize}
\pause
\begin{block}{Proposition}
    Bayesian modeling starts when fit-predict is useless
\end{block}

\pause
\begin{itemize}
    \item You have interpretable confidence intervals
    \item You have flexibility and control over the model
    \item You have reliable low data applications
\end{itemize}
\begin{alertblock}{It comes with a price...}
You ultimately HAVE to understand the model
\end{alertblock}
\end{frame}
\section{Supplementary}
\subsection{Programming}
\begin{frame}{PyMC}
\begin{columns}
\begin{column}{0.7\linewidth}
\begin{enumerate}
    \item Pure Python!
    \item Automated inference
    \item No complicated formulas for MCMC!
    \item Visualizations with ArviZ
    \item Reproducible research
    \item Used in industry applications
    \item Huge community
    \item Active development
\end{enumerate}
\url{https://github.com/pymc-devs/pymc}
\end{column}
\begin{column}{0.3\linewidth}
\includegraphics[width=\linewidth]{img/pymc-logo.png}
\includegraphics[width=\linewidth]{img/arviz-logo.png}
\includegraphics[width=\linewidth]{img/python-logo.png}
\end{column}
\end{columns}
\end{frame}
\begin{frame}[allowframebreaks]
\frametitle{References}
\bibliographystyle{abbrv}
\bibliography{references.bib}
\end{frame}

\end{document}

\documentclass{beamer}
\usepackage[OT2,T2A]{fontenc}
\usepackage[utf8]{inputenc}
\usepackage[russian,english]{babel}

\usepackage{ferres}
\framelogo{img/msu-logo}
%math
\input{math_commands.tex}
\author{Максим Кочуров}
\title[Прикладные байесовские методы]{Гауссовские процессы. Часть 2}
\institute[МГУ]{МГУ им. М.В. Ломоносова}
\date{Лекция 5}
\begin{document}
\begin{frame}
	\maketitle
\end{frame}
\begin{frame}{Содержание}
\tableofcontents
\end{frame}
\section{Введение}
\begin{frame}{Временные ряды, классический подход}
\begin{columns}
    \begin{column}{0.5\linewidth}
    Если в данных есть сезонность, обычно используют \href{https://www.statsmodels.org/devel/examples/notebooks/generated/stl_decomposition.html}{STL} декомпозицию (``\textbf{S}easonal and \textbf{T}rend decomposition using \textbf{L}oess''). Однако:
    %If data has seasonality, you usually use \href{https://www.statsmodels.org/devel/examples/notebooks/generated/stl_decomposition.html}{STL} decomposition. However,
    \begin{itemize}
        \item Параметры не интерпретируемы, есть только декомпозиция%Parameters are not interpretable, only decomposition is available
        \item Нет оценок неопределённости%No uncertainty estimates
        %huh? \item Quite strict on input values
        \item Недостаточная гибкость%Significantly less flexible in modelling
    \end{itemize}
    \end{column}
    \begin{column}{0.5\linewidth}
    \begin{figure}
        \centering
        \includegraphics[width=\linewidth]{img/examples_notebooks_generated_stl_decomposition_6_0}
        \caption{STL декомпозиция для данных об уровне CO2, Statsmodels}%STL decomposition for CO2 data, Statsmodels}
    \end{figure}
    \end{column}
\end{columns}
\end{frame}
\section{Подход с гауссовскими процессами}
\subsection{Введение}
\begin{frame}{Декомпозиция с гауссовскими процессами}
Гауссовские процессы поддерживают дополнительные по сравнению с STL предположения
%A Gaussian process can handle a complicated set of assumptions in addition to what STL provides
\begin{itemize}
    \item Детализация сезонности (год + квартал + месяц + неделя)%Granular seasonality (year + quarter + month + week)
    \item Модели с точками излома (changepoint). %Changepoint models
    \item Гибкая функция правдоподобия %Flexible likelihood Function
    \item Модели панельной регрессии %Panel regression models
    \item Пропущенные значения %Missing values
\end{itemize}
\end{frame}

\begin{frame}{Типичная модель}
Типичная модель аддитивная %Typical model is additive
    \begin{equation*}
        x_t \sim \underbrace{g(t)}_{\text{тренд}} + \underbrace{s(t)}_{\text{периодичность}} + \underbrace{h(t)}_{\text{праздники}}
    \end{equation*}
\begin{block}{Источник}
    См. \href{https://doi.org/10.7287/peerj.preprints.3190v2}{препринт} о моделях Prophet \cite{prophet_github}. Каждая модель временного ряда уникальна.
\end{block}
\end{frame}
\begin{frame}{Напоминание}
$x \in \R^n$, $y\in\R$
    \begin{align*}
    Y &\sim \mathcal{GP}(\alert<3>{m(x)}, \alert<4>{k(x, x')})\\
    \visible<2->{\begin{bmatrix} y_1 \\ \vdots \\ y_N \\ \end{bmatrix} &\sim
\mathcal{N}\left(
  \alert<3>{\begin{bmatrix} m(x_1)  \\\vdots\\ m(x_N)    \\ \end{bmatrix}} \,,
  \alert<4>{\begin{bmatrix} k(x_1,x_1)    & \dots & k(x_1, x_N)    \\
                  \vdots & \ddots& \vdots \\
                  k(x_N, x_1) & \dots & k(x_N, x_N)  \\ \end{bmatrix}}
        \right) \,}
    \end{align*}
    \begin{enumerate}
        \item<2-> $\mathcal{GP}$ гауссовский процесс -- многомерное нормальное распределение с матожиданием $m(t)$ и ковариацией $k(x, x')$ %Gaussian Process - simply, a normal distribution with special mean $m(x)$ and covariance $k(x, x')$
        \item<3-|alert@3> $m(x)$ -- функция среднего %mean function, e.g.
        \begin{itemize}
            \item Линейная регрессия: $m(x) = x^\top \beta$
            \item Константа (в т.ч. ноль): $m(x) = c$
            \item Произвольная функция: $m(x) = \sin(x)$
        \end{itemize}
        \item<4-|alert@4> $k(x, x')$ -- ядерная функция, мера сходства между $x$ и $x'$ %kernel function, simply - measure of similarity for $x$ and $x'$
        \begin{itemize}
            \item $[K]_{ij}=k(x_i, x_j)$ симметричная положительно определённая матрица %is an SPD matrix
        \end{itemize}
    \end{enumerate}
\end{frame}
\subsection{Непериодическая часть}
\begin{frame}{Непериодическая часть (среднее)}
\begin{columns}
    \begin{column}{0.5\linewidth}
\begin{itemize}
    \item<alert@2> Модели роста%Growth models
    \item<alert@3> Модели линейного тренда%Linear trend models
    \item<alert@4> Модели с точками излома (changepoint) %Changepoint models
\end{itemize}
\visible<5>{
\begin{block}{Расширения}
    Возможны расширения, например, зависящее от времени насыщение в модели роста, см. \cite{prophet_github}.
    %Extensions are possible, e.g. time dependent saturation in the growth model. See in \cite{prophet_github}
\end{block}
}
    \end{column}
    \begin{column}{0.5\linewidth}
    \only<2>{
        \begin{figure}
            \centering
            \includegraphics[width=\linewidth]{img/growth_model}
            \caption{Модель роста}
        \end{figure}
        \begin{equation*}
            x = \frac{c}{1 + \exp(-k(t-m))}
        \end{equation*}
    }
    \only<3>{
        \begin{figure}
            \centering
            \includegraphics[width=\linewidth]{img/linear_trend}
            \caption{Модель линейного тренда}
        \end{figure}
        \begin{equation*}
            x = \frac{c}{1 + \exp(-k(t-m))}
        \end{equation*}
    }
    \only<4->{
        \begin{figure}
            \centering
            \includegraphics[width=\linewidth]{img/changepoint_model}
            \caption{Модель с точкой излома}
        \end{figure}
        \begin{equation*}
            x = \begin{cases}
                c_1, \quad&t<m\\
                c_2, \quad&t\ge m
            \end{cases}
        \end{equation*}
    }
    \end{column}
\end{columns}
\end{frame}
\begin{frame}{Праздники}
 \begin{equation*}
        h(t) = \operatorname{is-holiday}(t)
    \end{equation*}
    \begin{figure}
        \centering
        \includegraphics[width=0.9\linewidth]{img/holidays}
        \caption{Признаки выходных дней}
    \end{figure}
\end{frame}
\subsection{Периодическая часть}
\begin{frame}{Периодическая часть (ковариация)}
    \begin{columns}
        \begin{column}{0.5\linewidth}
        Здесь важна степень детализации. Можно использовать несколько периодических ядер.
        %Granularities are important here. Multiple Periodic kernels can be used.
            \begin{itemize}
                \item год
                \item квартал
                \item месяц
                \item неделя
            \end{itemize}
        \end{column}
        \begin{column}{0.5\linewidth}
        \begin{figure}
            \centering
            \includegraphics[width=0.75\linewidth]{img/decomposed_time_series}
            \caption{Сезонная декомпозиция}
        \end{figure}
        \end{column}
    \end{columns}
\end{frame}
\begin{frame}{Масштабы длины для периодической части}
    \begin{block}{Гиперпараметры}
        Выбор масштаба длины на основе здравого смысла
        %Common sense driven lenthscale choice
    \end{block}
    \begin{itemize}
        \item Неделя -- динамика может измениться за пару дней ($ls\approx3$)%Week - couple of days make a change ($ls\approx3$)
        \item Месяц -- динамика может измениться за неделю ($ls\approx7$)
        \item Квартал -- динамика может измениться за месяц ($ls\approx30$)
        \item Год -- динамика может измениться за квартал ($ls\approx90$)
    \end{itemize}
    \pause
    \begin{alertblock}{На практике}
        Вместо периодического ядра используют признаки Фурье (Fourier features).
        %Everybody is using Fourier features as a replacement for Periodic Kernel
    \end{alertblock}
\end{frame}
\subsection{Модель}
\begin{frame}{Итоговая модель}
\begin{align*}
    m(t) &= \underbrace{g(t)}_{\text{апериодичная часть}} + \underbrace{h(t)}_{\text{выходные}}\\
    k(*, *) &= \visible<2->{\alert<2>{\alpha_{365}}}\operatorname{Periodic}(\rp=365, \rl=90)\\
      &+ \visible<2->{\alert<2>{\alpha_{90}}}\operatorname{Periodic}(\rp=90, \rl=30)\\
      &+ \visible<2->{\alert<2>{\alpha_{30}}}\operatorname{Periodic}(\rp=30, \rl=7)\\
      &+ \visible<2->{\alert<2>{\alpha_{7}}}\operatorname{Periodic}(\rp=7, \rl=3)\\
      &\visible<3->{\alert<3>{+ \beta \operatorname{ExpQuad}(\rl=90) +\operatorname{WhiteNoise}(\gamma)}}
\end{align*}
\visible<2->{
\begin{block}{Чего не хватает}
\begin{enumerate}
    \item<2-> Веса при периодических компонентах%Weights for periodic components
    \item<3-> Отклонения от тренда $g(t)$%Trent violations from $g(t)$
\end{enumerate}
\end{block}
}
\end{frame}
\subsection{Признаки Фурье}
\begin{frame}{Более эффективная периодичность}
\begin{columns}
    \begin{column}{0.5\linewidth}
    Некоторые оптимизации недоступны, когда в модели есть хотя бы одно периодичное ядро.
    %Having at least one $\operatorname{Periodic}$ kernel in a large time series prevents from optimizations.
    \begin{itemize}
        \item Можно добавить признаки Фурье как регрессоры%Fourier features can be added as regressors
        \item Это позволяет добавить в модель разумную периодичность%This allows reasonable periodicity to be present in the model
    \end{itemize}
    \end{column}
    \begin{column}{0.5\linewidth}
        \includegraphics[width=\linewidth]{img/fourier-features.png}
    \end{column}
\end{columns}
\end{frame}
\begin{frame}{Выбор признаков}
\begin{columns}
    \begin{column}{0.5\linewidth}
    У каждого периода есть оптимальное число компонент Фурье (опрядок)
    %Every regular period has optimal number of fourier components (aka order)
    \begin{itemize}
        \item неделя: 3
        \item месяц: 10
        \item год: 5
    \end{itemize}
    \end{column}
    \begin{column}{0.5\linewidth}
        \includegraphics[width=\linewidth]{img/periods-order.png}
    \end{column}
\end{columns}
\end{frame}
\begin{frame}{Применение признаков Фурье}
    \begin{itemize}
        \item Модели типа Prophet
        \item \emph{Очень} быстрая и доступная альтертатива гауссовским процессам%It is VERY Fast, drop-in replacement for Periodic GP
        \item Полезны для байесовских моделей%Proven to be useful for Bayesian models
    \end{itemize}
    \includegraphics[width=\linewidth]{img/ff-posterior-predictive.png}
\end{frame}
\begin{frame}{Интеграция в модель}
    \includegraphics[width=\linewidth]{img/fourier-features-code.png}
\end{frame}
\include{stoch-vol}
\begin{frame}[allowframebreaks]
\frametitle{Библиография}
\bibliographystyle{abbrv}
\bibliography{references.bib}
\end{frame}
\end{document}
